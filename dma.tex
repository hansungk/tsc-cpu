\documentclass[12pt]{article}
\usepackage[margin=1in]{geometry}
\usepackage[parfill]{parskip}
\usepackage{newtxtext}
\usepackage{booktabs}
\usepackage{fancyvrb}

\title{DMA \\
\large 18-1 ECE 322 Computer Organization, Lab 08}
\author{Hansung Kim \\ 2014-16824}
\date{}

\begin{document}
\maketitle

\section{Introduction}
Implement a DMA controller on top of the previous CPU implementation.
Learn how CPU cooperates with DMA to achieve better performance.

\section{Design}

The key point of this design is the use of a signal called
\verb|d_cache_busy| that indicates whether the cache is occupying the
data and address bus, i.e. handling cache miss.  The addition of this
single signal simplifies the logic of granting of the bus to the DMA.

\subsection{Bus grant and reclaim}

The CPU can grant the bus access to the DMA whenever the cache is not
handling miss, i.e. \verb|d_cache_busy == 0|.  Conversely, if the DMA
stops requesting for the bus, it is guaranteed that the bus is vacant
and the CPU can reclaim the access immediately.  Therefore, the code
for the bus grant and reclaim logic is simply as follows:

\begin{figure}[ht]
  \centering
  \begin{BVerbatim}
  if (bus_request && !d_cache_busy) begin
     bus_granted <= 1;
  end

  if (!bus_request) begin
     bus_granted <= 0;
  end
  \end{BVerbatim}
  \caption{Bus grant and reclaim logic using \texttt{d\_cache\_busy}}
  \label{fig:bus-logic}
\end{figure}

where \verb|d_cache_busy| is produced by testing if the cache is being
read or write but its value is not ready, i.e.  \texttt{(d\_readC OR
  d\_writeC) \&\& !d\_readyC}.

\subsection{Cycle stealing}
The DMA cycle stealing of the CPU can be implemented easily with
little modification.  After each memory write of 4 words, the DMA sets
\verb|bus_request| to zero for exactly one cycle.  This causes the CPU
to retry any currently blocked memory operation in that single cycle.
If there were any, the operation will set \verb|d_cache_busy| on,
delaying the \verb|bus_granted| to go up until the operation is
finished.  If there were no blocked memory operation, the logic in
Figure \ref{fig:bus-logic} would set \verb|bus_granted| back on
immediately.  Therefore, the Figure \ref{fig:bus-logic} logic is
capable of handling cycle stealing by itself.

\subsection{Changes to the cache}
The cache is modified to not issue any memory read/write command when
the bus is granted.  Its ``ready'' state should also remain false for
the whole time span of bus grant.  Thus, the \verb|readM|,
\verb|writeM| and \verb|readyC| signal is additionally logical-ANDed
with \verb|!bus_granted|, which effectively no-ops the cache when the
bus is granted to the DMA.

\subsection{Changes to the hazard unit}
The hazard unit stalls the entire pipeline on \verb|d_cache_busy|.
This single check remains to be enough for handling both the D-cache
miss and DMA bus block, as \verb|d_cache_busy| will be set to zero in
both cases.  Thus, the hazard unit does not require any modification
from the cache implementation (aside from the variable renaming).

\section{Implementation}
\section{Discussion}
\section{Conclusion}

\end{document}