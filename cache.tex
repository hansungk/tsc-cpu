\documentclass[12pt]{article}
\usepackage[margin=1in]{geometry}
\usepackage[parfill]{parskip}
\usepackage{newtxtext}
\usepackage{booktabs}
\usepackage{fancyvrb}

\title{Cache \\
\large 18-1 ECE 322 Computer Organization, Lab 07}
\author{Hansung Kim \\ 2014-16824}
\date{}

\begin{document}
\maketitle

\section{Introduction}
Implement cache on top of the previous pipelined CPU implementation.
Evaluate the speedup achieved by using cache by comparing no cache
version CPU with the new CPU.

\section{Design}
Several design points are:

\begin{itemize}
\item Write-through, write-no-allocate policy
\item Separate cache for instruction and data, both 16-word,
  direct-mapped
\item Memory with an 4 word read, 4 word write inout port (no
  single-word write)
\item inputReady-based asynchronous I/O signal model that supports
  variable latency
\item Parameterized module that can switch between cached/baseline
  model easily
\end{itemize}

\subsection{Write miss handling}

The multi-word write of the memory has some implication on the
write-miss performance of the cache: it should read the whole cache
block that contains the data to be written, before it can write to the
memory.  As a result, a write-miss takes 12 cycles which is a big
performance hit.  This can be improved if the memory supports a
single-word write.  (This was not done because I was unaware that the
specification allows memory to have separate data size for read and
write.)

After writing to the memory using a consecutive memory read and write,
no further operation is needed because write-no-allocate policy is
used for the write miss policy.

\subsection{Stall logic} \label{sec:stall-logic}

The ``safe'' stall logic mentioned above is devised because of the
probability that a control hazard can occur during a read-miss of the
instruction cache.  For I-cache miss, only the IF stage is stalled and
the ID and later stages is designed to carry on.  However, if the
hazard unit detects a wrong branch prediction during the IF stall, it
fails to reset PC because \verb|pc_write| has to remain zero in order
to stall IF.

\begin{figure}[ht]
  \centering
  \begin{BVerbatim}
    i1: IF ID EX MEM WB
    i2:    IF IF IF  ...
           ^  ^
           control hazard of i1 may happen here, but
           IF of i2 does not respond due to read-miss stall
  \end{BVerbatim}
  \caption{Illustration of control hazard during I-cache read miss}
\end{figure}

This can be easily solved by just staling the entire pipeline for a
read miss and thus preserving the control hazard signals until the
read miss resolves.  However, because of the bug that this change
uncovered in the baseline implementation in the last minute of the
deadline, the ``stupid'' version of stall logic is retained.

\section{Implementation}

A new module for the cache is implemented in \verb|cache.v|.  It is
connected with datapath and control units in \verb|cpu.v|.

Several implementation details and the performance measurement are
presented.

\subsection{I/O signaling} \label{sec:signal}

To accommodate for different memory latency for baseline and cache
implementation without changing logic, the input\_ready signal is used
to signify the master that the operation is finished and ready on the
inout port.  As soon as input\_ready signal rises, the master resets
the readM/writeM signal to zero, making the memory ready for the next
operation.  This is shown in \verb|memory.v| and \verb|cache.v|.

\subsection{Baseline implementation}

The implementation can be simply switched between cache-enabled or
baseline by changing the parameter called \verb|CACHE| given to the
cpu module and the memory module in \verb|cpu_TB.v|.

To minimize code change for cache and baseline implementation, the
baseline is implemented by modifying the cache to simply forward cache
read/write operation to the memory, and the cache data inout port to
the memory inout port.  This way we can reuse the I/O signaling logic
implemented in the cache (section \ref{sec:signal}), e.g. waiting for
input\_ready to be on, for the baseline as well.

\subsection{Measurements}

All measurements are done under the 2-bit hysteresis counter branch
prediction and data forwarding from the previous design enabled.

The performance comparison between the cache and baseline
implementation under the given workload is presented in Table
\ref{table:perf}.

\begin{table}[ht]
  \centering
  \begin{tabular}{@{}lrr@{}} \toprule
    Implementation & \# of cycles & IPC \\ \midrule
    Baseline & 4102 & 0.2392 \\
    Cache & 2939 & 0.3338 \\ \bottomrule
  \end{tabular}
  \caption{Performance of the cache-enabled and baseline implementation}
  \label{table:perf}
\end{table}

The speedup of the cache version over the baseline is 1.396.

The reason why the baseline exhibited such a high number of cycles is
due to the use of relaxed stall condition that stalls 2 more cycles
for every instruction read (section \ref{sec:stall-logic}).  This
rather extreme logic is used because a bug in the stall condition,
causing memory read and write signal to `race' under load and store
instructions, could not be fixed in time.  This affects both the cache
and baseline implementation, but because the baseline issues memory
read for every instruction, the performance impact is bigger for the
baseline.

The hit/miss count and ratio for the given workload is presented in
Table \ref{table:hitmiss}.

\begin{table}[ht]
  \centering
  \begin{tabular}{@{}lrrrr@{}} \toprule
    Cache & Hit & Miss & Total & Hit ratio \\ \midrule
    Instruction cache & 1391 & 159 & 1550 & 0.897 \\
    Data cache & 234 & 12 & 246 & 0.951 \\ \midrule
    Total & 1625 & 171 & 1796 & 0.905 \\
    \bottomrule
  \end{tabular}
  \caption{Hit ratio for both I-cache and D-cache}
  \label{table:hitmiss}
\end{table}

The instruction cache exhibits notably higher cache hit ratio.  This
can be attributed to the better spatial locality of the instruction
memory over the data memory, and that the execution time of the
Fibonacci loop takes the major part of the total execution time.  A
loop also exhibits high temporal locality, as well as high spatial
locality, because it reads the same instruction repeatedly over time.

\section{Discussion}
The implementation was successful and speedup from the cache could be
measured.  However, the inefficient stall logic and the choice of
multi-word write port for the memory negatively affected the
performance.  If the stall logic had been modified to eliminate the
unconditional 2-cycle stall for every instruction read, the baseline
implementation (currently taking 4 cycles per instruction) would have
been sped up greatly.  Considering the cache version would not see as
much performance increase due to fewer instruction memory access, the
speedup would not have been as high.  On the other hand, if the memory
was revised to support single-word write, it would make the additional
memory read in each cache write miss unnecessary, speeding up the
cache version.  Overall, the measured speedup value may not be a good
representation of the benefit coming solely from cache, for the given
workload.

The measurement showed high cache hit ratio of 90\% which does
indicate the cache version can potentially greatly benefit performance
for workloads that have better spatial and temporal locality.

\section{Conclusion}
Successfully implemented cache on top of the previous pipelined CPU
implementation.  Evaluated speedup achieved by using cache by
comparing no cache version CPU with the new CPU.

\end{document}